\input{preamble}
\usepackage{titlesec}
\usepackage[many]{tcolorbox}

% Adjust spacing after the chapter title
\titlespacing*{\chapter}{0cm}{-2.0cm}{0.50cm}
\titlespacing*{\section}{0cm}{0.50cm}{0.25cm}

% Indent 
\setlength{\parindent}{0pt}
\setlength{\parskip}{0.6ex}

% --- Theorems, lemma, corollary, postulate, definition ---
% \numberwithin{equation}{section}

\newtcbtheorem[]{problem}{Problem}%
    {enhanced,
    colback = white,
    colbacktitle = white,
    coltitle = black,
    boxrule = 0pt,
    frame hidden,
    fonttitle = \bfseries\sffamily,
    breakable,
    before skip = 1ex,
    after skip = 1ex
}{problem}

\tcbuselibrary{skins, breakable}

% --- You can define your own color box. Just copy the previous \newtcbtheorm definition and use the colors of yout liking and the title you want to use.
\input{commands}
\begin{document}

\noindent
\begin{minipage}[t]{0.55\textwidth}
    \textsf{\textbf{Student:}} Md Mobashir Rahman \\
    \textbf{Email:} \href{mailto:mdra00001.stud.uni-saarland.de}{\texttt{mdra00001.stud.uni-saarland.de}} \\
    \textsf{\textbf{Matriculation:}} 7059086
\end{minipage}
\hfill
\begin{minipage}[t]{0.4\textwidth}
    \raggedleft
    \begin{Large}
        \textsf{\textbf{Cellular Programs SS25}}\\
        Assignment 2
    \end{Large}
\end{minipage}


\vspace{2ex}















\begin{problem}{}{problem-label}
        

The unlabeled top row is simply the \textbf{“merge”}—a composite image in which the three single‑channel pictures underneath (cyan=Cyclin D1, yellow=MYB, magenta=Cyclin D1) are overlaid in false color. The merged view lets us see at a glance, how the three signals coexist in the very same cell at each differentiation stage (precursor→I→II→III→IV).

These changes in immunofluorescence color in the composite row show one protein switching off (Cyclin D1), another switching on then off (MYB), and the centrioles expanding (CEP43 marks centriole; when CEP43 is stained, each centriole lights up as a bright dot), all in a set order, reinforcing the paper’s claim that these post‑mitotic cells traverse a cell‑cycle‑like program rather than staying static.

\end{problem}



















\begin{problem}{}{problem-label-2}


Panel 2d plots the fraction of airway epithelial cells that become fully multiciliated after 5 days in culture. Vehicle‑treated wells (DMSO, grey dots) generate ~13–20\% multiciliated cells. Adding the highly selective CDK4/6 inhibitors palbociclib (blue) or ribociclib (pink) cuts that fraction to only ~5\%.

Because cell density is unchanged, \textbf{the drop can only reflect failed differentiation}—not toxicity or proliferation effects—so the graph directly demonstrates that active CDK4/6 is required to launch the multiciliated‑cell programme.

CDK4/6 needs its G1 cyclin partner \textbf{cyclin D1}.
The paper shows that supplying cyclin D1 fused to GFP increases the proportion of multiciliated cells, meaning cyclin D1 is an upstream trigger for the differentiation cascade. This sufficiency experiment is presented in \textbf{Fig.2l–m} (cyclin D1‑GFP gain‑of‑function) and summarized schematically in \textbf{Fig.2n}.

\end{problem}





















\begin{problem}{}{problem-label-3}

The canonical cell cycle involves distinct phases, including DNA synthesis (S phase) and mitosis (M phase), leading to cell division. To ascertain whether multiciliated cells complete this cycle, markers specific to these phases were examined.

 \textbf{Assessment of DNA Synthesis (S phase):} Extended Data Fig. 4a,b and Fig. 3h,i employed EdU incorporation during active DNA synthesis. Results show minimal EdU incorporation in wild-type ($E2f7^{+/+}$) multiciliated cells (identified by FOXJ1 expression), indicating a lack of S phase. Fig. 3h,i further demonstrates that loss of E2F7 ($E2f7^{-/-}$) leads to aberrant EdU incorporation, confirming that DNA synthesis is actively suppressed during normal multiciliation.

 \textbf{Assessment of Mitosis (M phase):} Extended Data Fig. 4c-e analyzed H3S10P, a marker for mitosis. A significant decrease in H3S10P-positive cells was observed as differentiation commenced.

These experiments provide evidence that differentiating multiciliated cells are post-mitotic. While they utilize a modified "multiciliation cycle" involving cell cycle regulators for differentiation processes like centriole amplification, they bypass DNA replication and mitosis, and therefore do not divide.

\end{problem}


























\begin{problem}{}{problem-label-4}




The paper demonstrates that during the ``multiciliation cycle'' in differentiating post-mitotic cells, \textit{E2F7} expression is elevated. Its crucial function is to actively repress the expression of genes required for DNA replication and thereby prevent aberrant DNA synthesis in these normally non-dividing cells. The experiments show that loss of \textit{E2F7} (\textit{E2f7}$^{-/-}$) leads to the inappropriate initiation of DNA synthesis (indicated by increased EdU incorporation) in these multiciliated cells.

Tumor growth is fundamentally driven by uncontrolled cell proliferation, which needs repeated rounds of DNA synthesis and cell division. Therefore, if \textit{E2F7} acts as a brake on DNA synthesis (as shown in the specific context of the Paper), \textbf{enhancing its function or upregulation} \textbf{(option C)} would theoretically strengthen this braking mechanism. This could potentially slow down or halt the DNA replication required for tumor cell proliferation.

\end{problem}


% =================================================

% \newpage

% \vfill


\end{document}